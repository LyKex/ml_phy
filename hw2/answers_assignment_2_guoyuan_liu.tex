\documentclass[11pt]{article}
\usepackage{amsmath,amsthm,amsfonts,amssymb,amscd}
\usepackage{multirow,booktabs}
\usepackage[table]{xcolor}
\usepackage{fullpage}
\usepackage{lastpage}
\usepackage{enumitem}
\usepackage{fancyhdr}
\usepackage{mathrsfs}
\usepackage{wrapfig}
\usepackage{setspace}
\usepackage{calc}
\usepackage{multicol}
\usepackage{cancel}
\usepackage[retainorgcmds]{IEEEtrantools}
\usepackage[margin=3cm]{geometry}
\usepackage{amsmath}
\newlength{\tabcont}
\setlength{\parindent}{0.0in}
\setlength{\parskip}{0.05in}
\usepackage{empheq}
\usepackage{framed}
\usepackage[most]{tcolorbox}
\usepackage{xcolor}
\colorlet{shadecolor}{orange!15}
\parindent 0in
\parskip 12pt
\geometry{margin=1in, headsep=0.25in}
\theoremstyle{definition}
\newtheorem{defn}{Definition}
\newtheorem{reg}{Rule}
\newtheorem{exer}{Exercise}
\newtheorem{note}{Note}
\begin{document}
\title{Chapter 9 Review Notes}
\thispagestyle{empty}
\begin{center}
{\LARGE \bf Homework 2}\\
{\large Guoyuan Liu}\\
Fall 2023, PHYS-467
\end{center}

\section{Question 2}
Notice $P(G|\{g_i\}, \theta)$ is similar to binomial distribution with the probability of success, or the existence of an edge, as a variable. So we can write
\begin{equation}
    P(G|\{g_i\}, \theta) = \prod_{i\neq j}^N p_{g_i,g_j}^{A_{ij}} (1 - p_{g_i , g_j}^{A_{ij}})^ (1-A_{ij})
\end{equation}
and the assignment of $g_i$ is just
\begin{equation}
    P(\{g_i\}|\theta) = \prod_i^N n_{g_i}
\end{equation}
With Bayes' theorem, we have
\begin{align}
        P(\{g_i\}| G, \theta) = \frac{P(G|\{g_i\}, \theta) P(\{g_i\}|\theta)}{\sum_{\{g_i\}} P(G|\theta)}
\end{align}
\end{document}